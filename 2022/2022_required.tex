\documentclass{jsarticle}
\usepackage[dvipdfmx]{graphicx}
\usepackage[dvipdfmx]{color}
\usepackage{amsmath, amsthm, amscd, amsfonts, amssymb}
\usepackage{ascmac}
\usepackage{mathrsfs}
\usepackage{url}
\usepackage{listings}
\usepackage{enumerate}
\usepackage{latexsym}
\usepackage{bm}
\usepackage{tikz}
\usepackage{physics}
\usepackage{fancyhdr}
\pagestyle{fancy}

\newcommand{\examYear}{2022}
\newcommand{\answerCreater}{glitter}

\title{電気通信大学\\ 情報・ネットワーク工学専攻(I専攻)\\ \examYear 年度入学試験 \\ 専門科目: [必須問題] 自作解答}
\author{解答作成者: \answerCreater}
\date{最終更新日: \today}

\lstset{
  basicstyle={\ttfamily},
  identifierstyle={\small},
  commentstyle={\smallitshape},
  keywordstyle={\small\bfseries},
  ndkeywordstyle={\small},
  stringstyle={\small\ttfamily},
  frame={tb},
  breaklines=true,
  columns=[l]{fullflexible},
  numbers=left,
  xrightmargin=0zw,
  xleftmargin=3zw,
  numberstyle={\scriptsize},
  stepnumber=1,
  numbersep=1zw,
  lineskip=-0.5ex
}

\begin{document}
\maketitle
\begin{abstract}
  
\end{abstract}
\tableofcontents
\lhead{電気通信大学 I専攻 \examYear 年度入試 専門科目: [必須問題] 自作解答}
\cfoot{\thepage}

\newpage
\section{線形代数}
\begin{enumerate}[(1)]
  \item
  \begin{align*}
    |A- \lambda E| = 
    \begin{vmatrix}
      -1 -\lambda & 1 + a & -2 \\
      0 & 1 - \lambda & 0 \\
      4 & 1-a & 5 -\lambda \\
    \end{vmatrix}
    = -
    \begin{vmatrix}
      1-\lambda & 0 \\
      1-a & 5 -\lambda \\
    \end{vmatrix}
    + 4
    \begin{vmatrix}
      1 + a & -2 \\
      1 -\lambda & 0 \\
    \end{vmatrix}
  \end{align*}

  \item (1)より,$\lambda_1 = 3$.$A-\lambda_1 E$の有限回の基本変形によって,
  \begin{align*}
    A -\lambda_1 E
  \end{align*}
  
  が得られるので,

  \item (1)より,$\lambda_2 = 1$である.$(\lambda_2 E - A)^2$の有限回の基本変形によって,
  \begin{align*}
    (\lambda_2 E- A)^2 =
    \begin{matrix}
      2 & -1-a & 2 \\
      0 & 0 & 0 \\
      -4 & a -1 & -4 \\
    \end{matrix}^2
    =
    \begin{matrix}
      -4 & -4 & -4 \\
       0 & 0 & 0 \\
       8 & 8 & 8 \\
    \end{matrix}
  \end{align*}

  を得るから,$\dim\ker f=\;,\dim\Im f = $である.

  \item 
\end{enumerate}

\newpage
\section{微分積分}
\begin{enumerate}[(1)]
  \item 
  \begin{align*}
    f_x(0,0) &= \\
    f_y(0,0) &= \\
    f_{xx}(0,0) &= \\
    f_{xy}(0,0) &= \\
    f_{yy}(0,0) &= \\
  \end{align*}

  \item 
  \begin{align*}
    g_x &= (2x+y)e^y \\
    g_y &= (x^2 + xy + x)e^y=x(x + y + 1)e^y\\
    g_{xx} &= 2e^y\\
    g_{xy} &= (y + 1)e^y\\
    g_{yy} &= (x^2 + xy + 2x)e^y\\
  \end{align*}
  \item 
  \begin{enumerate}[(i)]
    \item $x=r\cos\theta, y=r\sin\theta$で,$D_1$は$\{(r,\theta)\in\mathbb{R}^2| 0\leq r\leq \pi,-\pi \leq \theta < \pi\}$と連続に1対1に対応する.
    \begin{align*}
      I_1 =   
    \end{align*}

    \item $u = x + y, v = x-y$で,$D_2$は$\{(u,v)\in\mathbb{R}^2\}$
  \end{enumerate}
\end{enumerate}


\end{document}
