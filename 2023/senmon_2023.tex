\documentclass{jsarticle}
\usepackage[dvipdfmx]{graphicx}
\usepackage{amsthm}
\usepackage{amscd}
\usepackage{ascmac}
\usepackage{mathrsfs}
\usepackage{url}
\usepackage{listings}
\usepackage{enumerate}
\usepackage{latexsym}
\usepackage{amsfonts}
\usepackage{amssymb}
\usepackage{bm}
\usepackage{tikz}
\usepackage{amsmath}
\usepackage{newunicodechar}
\usepackage{physics}
\theoremstyle{definition}
\parindent = 0pt
\begin{document}
\title{電通大 2023年度 専門科目:[選択問題]}
\author{解答作成者:glitter}
\date{最終更新日:\today}
\maketitle
  \begin{abstract}
    第7問の「数値計算」では、ヤコビ法による解の収束性について出題された。問題に沿って解答していけば、特に複雑な変形等は必要ない。
  \end{abstract}
  \section{電気回路}

  \section{電磁気学}

  \section{確率統計}

  \section{信号処理}

  \section{アルゴリズムとデータ構造}

  \section{計算機の基本原理}

  \newpage
  \section{数値計算}
  \begin{enumerate}[(a)]
    \item
    \begin{align*}
      N = 
      \begin{bmatrix}
        1/a_{11} & & \\
        & \ddots & \\
        & & 1/a_{nn}
      \end{bmatrix}
      = \underline{D^{-1}}
    \end{align*}
    \begin{align*}
      M &=
      \begin{bmatrix}
        0 & \cdots & a_{1i}/a_{11}  & \cdots & a_{1n}/a_{11} \\
        \vdots & \ddots  & \vdots  & \ddots & \vdots \\
        a_{i1}/a_{ii} & \cdots  & 0  & \cdots & a_{in}/a_{ii} \\
        \vdots & \ddots &  \vdots & \ddots & \vdots \\
        a_{n1}/a_{nn} & \cdots & a_{ni}/a_{nn} & \cdots & 0 \\
      \end{bmatrix} \\
      &=
      \begin{bmatrix}
        1/a_{11} & & \\
        & \ddots & \\
        & & 1/a_{nn}
      \end{bmatrix}
      \begin{bmatrix}
        0 & \cdots & a_{1i} & \cdots & a_{1n} \\
        \vdots & \ddots & \vdots & \ddots & \vdots \\
        a_{i1} & \cdots & 0  & \cdots & a_{in} \\
        \vdots & \ddots & \vdots & \ddots & \vdots \\
        a_{n1} & \cdots & a_{ni} & \cdots & 0 \\
      \end{bmatrix}
      = \underline{-D^{-1}(L+U)}
    \end{align*}
    \item (a)より,
    \begin{math}
      N=
      \begin{bmatrix}
        1/4 & 0 \\
        0 & 1/2 \\
        \end{bmatrix}        
    \end{math},
    \begin{math}
      M= -
      \begin{bmatrix}
        1/4 & 0 \\
        0 & 1/2 \\
      \end{bmatrix}
      \left(
        \begin{bmatrix}
          0 & 0 \\
          1 & 0 \\
        \end{bmatrix}
        +
        \begin{bmatrix}
          0 & 1 \\
          0 & 0 \\
        \end{bmatrix}
      \right)
      = \underline{-
      \begin{bmatrix}
        0 & 1/4 \\
        1/2 & 0 \\
      \end{bmatrix} 
      }
    \end{math}

    \item 
    \begin{math}
      N
      \begin{bmatrix}
        15 \\ 9
      \end{bmatrix}
      = 
      \begin{bmatrix}
        15/4 \\ 9/2
      \end{bmatrix}
    \end{math}
    で,
    \begin{align*}
      \vb{x}^{(1)} &= -
      \begin{bmatrix}
        0 & 1/4 \\
        1/2 & 0 \\
      \end{bmatrix}
      \vb{x}^{(0)} +
      \begin{bmatrix}
        15/4 \\ 9/2
      \end{bmatrix}
      = 
      \begin{bmatrix}
        15/4 \\ 9/2
      \end{bmatrix}
      \\
      \vb{x}^{(2)} &= -
      \begin{bmatrix}
        0 & 1/4 \\
        1/2 & 0 \\
      \end{bmatrix}
      \begin{bmatrix}
        15/4 \\ 9/2
      \end{bmatrix}
      +
      \begin{bmatrix}
        15/4 \\ 9/2
      \end{bmatrix}
      = 
      \begin{bmatrix}
        21/8 \\ 21/8
      \end{bmatrix}
      \\
      \vb{x}^{(3)} &= -
      \begin{bmatrix}
        0 & 1/4 \\
        1/2 & 0 \\
      \end{bmatrix}
      \begin{bmatrix}
        21/8 \\ 21/8
      \end{bmatrix} 
      +
      \begin{bmatrix}
        15/4 \\ 9/2
      \end{bmatrix}
      = 
      \begin{bmatrix}
        99/32 \\ 51/16
      \end{bmatrix}
      \\
    \end{align*}

    \item 
    \begin{align*}
      \|A\vb{x}\| = \max_{1\leq i\leq n}\sum_{j=1}^n |a_{ij}x_j| \leq \max_{1\leq i\leq n}\sum_{j=1}^n |a_{ij}|\max_{1\leq k\leq n}|x_k| = \|A\|\|\vb{x}\|
    \end{align*}

    \item (a)の途中式より
    \begin{align*}
      \|M\| = \max_{1\leq i\leq n}\sum_{j=1, j\neq i}^n \left|\dfrac{a_{ij}}{a_{ii}}\right| = \max_{1\leq i\leq n}\dfrac{1}{|a_{ii}|}\sum_{j=1,j\neq i}|a_{ij}| < \max_{1\leq i\leq n}\dfrac{1}{|a_{ii}|} |a_{ii}| = 1
    \end{align*}
    \item $g(\vb{x}) =M\vb{x}+N\vb{b}$とすると,任意の$\vb{x}\in\mathbb{R}^n$に対して,$g(\vb{x})\in \mathbb{R}^n$(条件(i)).$\vb{x,y}\in\mathbb{R}^n$について,
    \begin{align*}
      \|g(\vb{x})-g(\vb{y})\| = \|(M\vb{x}+N\vb{b}) - (M\vb{y}+N\vb{b})\| = \|M(\vb{x-y})\|\leq \|M\|\|\vb{x-y}\|\;({\rm (d)より})
    \end{align*}
    で,(e)より$0\leq\|M\|< 1$であるから条件(ii),(iii)も成り立つ.よって,定理の条件を満たすため,ヤコビ法によって$g(\vb{x}) = \vb{x}\Leftrightarrow M\vb{x}+N\vb{b} = \vb{x}\Leftrightarrow A\vb{x} = \vb{b}$の解に収束する.
  \end{enumerate}
  \section{離散数学とオートマトン}
\end{document}
