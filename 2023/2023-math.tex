\documentclass[11pt, titlepage]{jsarticle}

\usepackage[dvipdfmx]{graphicx}
\usepackage[dvipdfmx]{color}
\usepackage{amsmath,amssymb}
\usepackage{listings}
\usepackage{physics}
\usepackage{pdfpages}
\usepackage{fancyhdr}
\pagestyle{fancy}

\title{電気通信大学院試 I専攻\\2023年度 専門科目: [必須問題]}
\author{回答作成者: Zerr}
\date{最終更新日: \today}

\lstset{
  basicstyle={\ttfamily},
  identifierstyle={\small},
  commentstyle={\smallitshape},
  keywordstyle={\small\bfseries},
  ndkeywordstyle={\small},
  stringstyle={\small\ttfamily},
  frame={tb},
  breaklines=true,
  columns=[l]{fullflexible},
  numbers=left,
  xrightmargin=0zw,
  xleftmargin=3zw,
  numberstyle={\scriptsize},
  stepnumber=1,
  numbersep=1zw,
  lineskip=-0.5ex
}
\renewcommand{\lstlistingname}{図}

\makeatletter
\AtBeginDocument{%
  \let\c@figure\c@lstlisting
  \let\thefigure\thelstlisting
  \let\ftype@lstlisting\ftype@figure
}
\makeatother

\begin{document}
\chead{電気通信大学院試 I専攻2023年度 必須問題 自作解答}
\cfoot{\thepage}

\maketitle
\section{線形代数}
\begin{eqnarray*}
  A&=&\begin{pmatrix}
    -4 & 2  & -1 & -7 \\
    0  & 5  & -3 & 0  \\
    0  & 4  & -3 & 0  \\
    3  & -2 & 1  & 6
  \end{pmatrix}\\
  \boldsymbol p_1&=&\begin{pmatrix}
    1 \\
    0 \\
    0 \\
    -1
  \end{pmatrix}
  ,\boldsymbol p_2=\begin{pmatrix}
    7 \\
    1 \\
    2 \\
    -3
  \end{pmatrix}
  ,\boldsymbol p_3=\begin{pmatrix}
    9 \\
    2 \\
    4 \\
    -1
  \end{pmatrix}
\end{eqnarray*}

\subsection*{(1)}
余因子展開とサラスの公式より、
\begin{eqnarray*}
  \det A = 
  \begin{vmatrix}
    -4 & 2  & -1 & -7 \\
    0  & 5  & -3 & 0  \\
    0  & 4  & -3 & 0  \\
    3  & -2 & 1  & 6
  \end{vmatrix}&=&
  \begin{vmatrix}
    -1 & 2  & -1 & -1 \\
    0  & 5  & -3 & 0  \\
    0  & 4  & -3 & 0  \\
    3  & -2 & 1  & 6
  \end{vmatrix}=
  \begin{vmatrix}
    -1 & 2  & -1 & -1 \\
    0  & 1  & 0  & 0  \\
    0  & 4  & -3 & 0  \\
    0  & -2 & 1  & 3
  \end{vmatrix}\\
  &=& (-1)
  \begin{vmatrix}
    1  & 0  & 0 \\
    4  & -3 & 0 \\
    -2 & 1  & 6
  \end{vmatrix}
  = (-1)(1\cdot-3\cdot6)=9
\end{eqnarray*}

\newpage
\subsection*{(2)}
$p_3\in V$であることは、ある$c_1, c_2\in \mathbb{R}$があって
\begin{eqnarray*}
  \boldsymbol p_3 = c_1 \boldsymbol p_1 + c_2 \boldsymbol p_2 \quad (c_1, c_2 \in \mathbb R)
\end{eqnarray*}
となることを示せばいい。これは
\begin{eqnarray*}
  &&\begin{pmatrix}
    \boldsymbol p_1 & \boldsymbol p_2 & -\boldsymbol p_3
  \end{pmatrix}\begin{pmatrix}
    c_1 \\
    c_2 \\
    1
  \end{pmatrix}=\boldsymbol 0
\end{eqnarray*}

という一次方程式系の階が存在することと同値。行列$(\boldsymbol p_1, \boldsymbol p_2, -\boldsymbol p_3)$の有限回の基本変形により、

\begin{eqnarray*}
  \left(\begin{array}{ccc|c}
    1  & 7  & -9 & 0 \\
    0  & 1  & -2 & 0 \\
    0  & 2  & -4 & 0 \\
    -1 & -3 & 1  & 0 \\
  \end{array}\right)&&
  \rightarrow\left(\begin{array}{ccc|c}
    1 & 7 & -9 & 0 \\
    0 & 1 & -2 & 0 \\
    0 & 2 & -4 & 0 \\
    0 & 4 & -8 & 0 \\
  \end{array}\right)
  \rightarrow\left(\begin{array}{ccc|c}
    1 & 0 & 5  & 0 \\
    0 & 1 & -2 & 0 \\
    0 & 0 & 0  & 0 \\
    0 & 0 & 0  & 0 \\
  \end{array}\right)
\end{eqnarray*}
となり
\begin{eqnarray*}
  \begin{pmatrix}
    c_1 \\
    c_2
  \end{pmatrix}=\begin{pmatrix}
    -5 \\
    2
  \end{pmatrix}
\end{eqnarray*}
が得られる。以上より、$p_3\in V$が示された。

\newpage
\subsection*{(3)}

\begin{eqnarray*}
  A \boldsymbol p_1&=&\begin{pmatrix}
    -4 & 2  & -1 & -7 \\
    0  & 5  & -3 & 0  \\
    0  & 4  & -3 & 0  \\
    3  & -2 & 1  & 6
  \end{pmatrix}\begin{pmatrix}
    1 \\
    0 \\
    0 \\
    -1
  \end{pmatrix}=\begin{pmatrix}
    3 \\
    0 \\
    0 \\
    -3
  \end{pmatrix}=3 \boldsymbol p_1\\
  A \boldsymbol p_2&=&\begin{pmatrix}
    -4 & 2  & -1 & -7 \\
    0  & 5  & -3 & 0  \\
    0  & 4  & -3 & 0  \\
    3  & -2 & 1  & 6
  \end{pmatrix}\begin{pmatrix}
    7 \\
    1 \\
    2 \\
    -3
  \end{pmatrix}=\begin{pmatrix}
    -7 \\
    -1 \\
    -2 \\
    3
  \end{pmatrix}=-1 \boldsymbol p_2\\
  A \boldsymbol p_3&=&\begin{pmatrix}
    -4 & 2  & -1 & -7 \\
    0  & 5  & -3 & 0  \\
    0  & 4  & -3 & 0  \\
    3  & -2 & 1  & 6
  \end{pmatrix}\begin{pmatrix}
    9 \\
    2 \\
    4 \\
    -1
  \end{pmatrix}=\begin{pmatrix}
    -29 \\
    -2  \\
    -4  \\
    21
  \end{pmatrix}
\end{eqnarray*}
より,$\boldsymbol p_1$には$\alpha_1=3$が,$\boldsymbol p_2$には$\alpha_2=-1$が条件

\begin{eqnarray*}
  A \boldsymbol p_i = \alpha_i \boldsymbol p_i
\end{eqnarray*}

を満たす実数となる.一方、$\boldsymbol p_3$は$A \boldsymbol p_3$をスカラー倍で表せないため$\alpha_3$は存在しない.実際、$\vb{p_3}$と$A\vb{p_3}$の各成分の比を考えると、第二成分($-1$)と第四成分($-21$)で一致しない。

\newpage
\subsection*{(4)}
(3)より、$\alpha=-1$である。これを固有値にもつ固有ベクトルを求める。有限回の基本変形で
\begin{eqnarray*}
  A-\alpha I&=&\begin{pmatrix}
    -3 & 2  & -1 & -7 \\
    0  & 6  & -3 & 0  \\
    0  & 4  & -2 & 0  \\
    3  & -2 & 1  & 7
  \end{pmatrix}\rightarrow\begin{pmatrix}
    -3 & 0 & 0  & -7 \\
    0  & 2 & -1 & 0  \\
    0  & 0 & 0  & 0  \\
    0  & 0 & 0  & 0
  \end{pmatrix}
\end{eqnarray*}

で得られるから、$(A-\alpha I)(x, y, z, w)^\top = \vb{0}$なる方程式の解は以下のようになる。
\begin{eqnarray*}
  \begin{pmatrix}
    x \\
    y \\
    z \\
    w
  \end{pmatrix}=\boldsymbol 0\\
  &&
  \left\{ \,
  \begin{aligned}
    x & =-\frac{7}{3}w                   \\
    y & =\frac{1}{2}z                    \\
    z & =s\quad (s\in\mathbb R, s\neq 0) \\
    w & =t\quad (t\in\mathbb R, t\neq 0)
  \end{aligned}
  \right.
\end{eqnarray*}
これより,Aの$\alpha=-1$に関する固有ベクトルで、一次独立なものとして$(0,1,2,0)$、$(7, 0, 0, -3)$が取れるため,求める固有空間の基底として
\begin{eqnarray*}
  \left\langle
  \begin{pmatrix}
    0 \\
    1 \\
    2 \\
    0
  \end{pmatrix},\begin{pmatrix}
    7 \\
    0 \\
    0 \\
    -3
  \end{pmatrix}
  \right\rangle
\end{eqnarray*}
がとれる。

\newpage
\subsection*{(5)}
$A^n$があるからといって,直ぐに$A^n$の対角化に取り掛かると固有ベクトルの数が1つ足りずに詰む為注意(1敗)

(2)より,$\boldsymbol p_3=-5 \boldsymbol p_1 + +2 \boldsymbol p_2$が成り立つため
\begin{eqnarray*}
  A^n \boldsymbol p_3 &=& -5 A^n \boldsymbol p_1 + 2 A^n \boldsymbol p_2 \\
  &=& -5\alpha_1^n p_1 + 2\alpha_2^n p_2 \\
  &=& -5\cdot(3)^n \begin{pmatrix}
    1 \\
    0 \\
    0 \\
    -1
  \end{pmatrix} + 2\cdot(-1)^n \begin{pmatrix}
    7 \\
    1 \\
    2 \\
    -3
  \end{pmatrix}\\
  &=&\begin{pmatrix}
    -5\cdot(3)^n  + 14\cdot(-1)^n \\
    2\cdot(-1)^n                  \\
    4\cdot(-1)^n                  \\
    5\cdot(3)^n  -6\cdot(-1)^n
  \end{pmatrix}
\end{eqnarray*}
となる.

\newpage
\section{微分積分}
\subsection*{(1)}
\begin{eqnarray*}
  f(x,y)&=&x^2+y^2\\
  g(x,y)&=&(x-y-1)^2+(2x+y-2)^2+19
\end{eqnarray*}
\subsubsection*{(i)}
\begin{eqnarray*}
  &\frac{\partial g}{\partial x}&=2(x-y-1)+2(2x+y-2)\cdot 2=10x+2y-10\\
  &\frac{\partial g}{\partial y}&=2(x-y-1)\cdot(-1)+2(2x+y-2)=2x+4x-2\\
  &\frac{\partial g}{\partial x}&(1,1)=2\\
  &\frac{\partial g}{\partial y}&(1,1)=4
\end{eqnarray*}
となるため,接平面は
\begin{eqnarray*}
  z-g(1,1)&=&2(x-1)+4(y-1) \\
\end{eqnarray*}
となる。整理して、

\begin{equation*}
  h(x,y) = 2x + 4y + 15
\end{equation*}

\newpage
\subsubsection*{(ii)}
曲面$z=f(x,y)$と平面$P$の共通部分$F$は以下の等式を満たす$(x, y, f(x,y))=(x,y,h(x,y))$の集まりであるので
\begin{eqnarray}
  &x^2+y^2=2x+4y+15 \notag\\
  \Leftrightarrow &x^2+y^2-2x-4x-15=0 \notag\\
  \Leftrightarrow &(x-1)^2 + (y-2)^2 = 20 \label{e_2_1}
\end{eqnarray}
$F$の最大となる$z$座標は(\ref{e_2_1})の条件下での$h(x,y)$の最大値である.

($l: \mathbb R^2 \rightarrow \mathbb R, (x,y)\mapsto (x-1)^2+(y-2)^2-20$とすると(\ref{e_2_1})は連続関数である.
$\left\{ 0 \right\}$は$\mathbb R$の閉集合なので,その逆像$f^{-1}(\{0\})$は$\mathbb R^2$の閉集合である.(\ref{e_2_1})は有界でもあるため,(\ref{e_2_1})は有界閉集合となる.)

ユークリッド空間では「有界閉集合である $\Leftrightarrow$ コンパクト集合である」から,$F$は
コンパクト集合であるため最大値と最小値を持つ.なのでラグランジュの未定乗数法を用いて(\ref{e_2_1})上で最大となる点の候補を求める.まず,ラグランジュ関数$L$は次のようになる.

\begin{eqnarray*}
  L(x,y,\lambda)=2x+4y+15 + \lambda\left\{ (x-1)^2 + (y-2)^2 - 20 \right\}
\end{eqnarray*}
各変数で偏微分を行い,この関数の極値点を求める.

\begin{eqnarray*}
  \frac{\partial L}{\partial x}&=&2+2\lambda(x-1)=0 \\
  &\Leftrightarrow& x-1=-\frac{1}{\lambda} \\
  \frac{\partial L}{\partial y}&=&4+2\lambda(y-2)=0 \\
  &\Leftrightarrow& y-2=-\frac{2}{\lambda} \\
  \frac{\partial L}{\partial \lambda}&=&(x-1)^2 + (y-2)^2-20=0 \\
  &\Leftrightarrow&(-\frac{1}{\lambda})^2 + (-\frac{2}{\lambda})^2-20=0 \\
  &\Leftrightarrow&20\lambda^2=5\\
  &\Leftrightarrow&\lambda=\pm \frac{1}{2}\\
\end{eqnarray*}

$\lambda=\pm \frac{1}{2}$より,解の候補は$(x,y)=\left( 1\pm2,  2\pm4 \right) (複号同順)$である.$h(3,6) = 45 > h(-1, -2) = 5$より,z座標が最大となる点は$\left(3, 6, 45\right)$である.

\newpage
\subsection*{(2)}
\subsubsection*{(i)}
\begin{eqnarray*}
  I&=&\int\int_D e^{y^2} dxdy, \quad D=\left\{ (x,y) | y^3 \leqq x \leqq y, 0 \leqq y \leqq 1 \right\} \\
  &=&\int_0^1 \int_{y^3}^y e^{y^2}dxdy \\
  &=&\int_0^1 ye^{y^2}dy - \int_0^1 y^2\left(ye^{y^2}\right) dy \\
  &=&\left[ \frac{1}{2}e^{y^2} \right]_0^1 - \left[y^2\frac{1}{2}e^{y^2} \right]_0^1 + \int_0^1 2y\cdot\frac{1}{2}e^{y^2} dy \\
  &=&-\frac{1}{2} + \int_0^1 ye^{y^2}dy=\left[ \frac{1}{2}e^{y^2} \right]_0^1-\frac{1}{2}=\frac{1}{2} \left( e-2 \right)
\end{eqnarray*}

\subsubsection*{(ii)}
\begin{eqnarray*}
  J&=&\int\int_E \frac{x}{x^2+y^2} dxdy, \quad D=\left\{ (x,y) | 1 \leqq x^2+y^2 \leqq 2x, y \geqq 0 \right\} \\
  x&=&rcos\theta, y=rsin\theta とおくと,Dは\\
  D'&=&\left\{ (x,y) | 1 \leqq r \leqq 2cos\theta, 0 \leqq \theta \leqq \pi \right\}に全単射で移る.このとき\\
  &&\frac{\partial (x,y)}{\partial (r,\theta)}=\begin{vmatrix}
    \frac{\partial x}{\partial r}=cos\theta & \frac{\partial x}{\partial \theta}=-rsin\theta \\
    \frac{\partial y}{\partial r}=sin\theta & \frac{\partial y}{\partial \theta}=rcos\theta
  \end{vmatrix}=r(sin^2\theta + cos^2\theta)=r \\
  よ&っ&て\\
  &=&\int_0^\pi \int_1^{2cos\theta} \frac{1}{r}cos\theta \cdot r drd\theta \\
  &=&\int_0^\pi \left[ rcos\theta \right]_1^{2cos\theta} d\theta\\
  &=&\int_0^\pi 2cos^2\theta - cos\theta d\theta \\
  &=&\int_0^\pi 2\left( \frac{cos2\theta + 1}{2} \right) - cos\theta d\theta \\
  &=&\left[ \frac{1}{2}sin2\theta + \theta - sin\theta \right]_0^\pi \\
  &=&\pi
\end{eqnarray*}

\newpage
\subsubsection*{(iii)}
\begin{eqnarray*}
  L_n&=&\sum_{j=1}^n \sum_{k=1}^n \frac{1}{nj+nk+n^2} \\
  &=&\frac{1}{n^2} \sum_{j=1}^n \sum_{k=1}^n \frac{1}{\frac{j}{n}+\frac{k}{n}+1} \\
  L&=& \lim_{n\rightarrow\infty} \frac{1}{n^2} \sum_{j=1}^n \sum_{k=1}^n \frac{1}{\frac{j}{n}+\frac{k}{n}+1}\\
  &=&\int_0^1\int_0^1 \frac{1}{x+y+1}dxdy\\
  &=&\int_0^1\left[ \log\left| x+y+1 \right| \right]_0^1 dy\\
  &=&\int_0^1 \log \left| y+2 \right| - \log \left| y+1 \right| dy\\
  &&y+2=u, dy=du,\quad y+1=v, dy=dv \\
  &=&\int_2^3 \log\left| u \right|du - \int_1^2 \log\left| v \right|dv\\
  &=&\left[ u\log\left| u \right| + u\right]_2^3 - \left[v\log\left| v \right| + v\right]_1^2 \\
  &=&3\log3 + 3 - (2\log2 + 2) - (2\log2 + 2) + 1\\
  &=&3\log3-4\log2
\end{eqnarray*}


\end{document}
