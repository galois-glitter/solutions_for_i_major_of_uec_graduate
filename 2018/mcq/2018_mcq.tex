\documentclass{jsarticle}
\usepackage[dvipdfmx]{graphicx}
\usepackage[dvipdfmx]{color}
\usepackage{amsmath, amsthm, amscd, amsfonts, amssymb}
\usepackage{ascmac}
\usepackage{mathrsfs}
\usepackage{url}
\usepackage{listings}
\usepackage{enumerate}
\usepackage{latexsym}
\usepackage{bm}
\usepackage{tikz}
\usepackage{circuitikz}
\usepackage{physics}
\usepackage{fancyhdr}
\pagestyle{fancy}

\newcommand{\examYear}{2018}
\newcommand{\answerCreater}{}

\title{電気通信大学\\ 情報・ネットワーク工学専攻(I専攻)\\ \examYear 年度入学試験 \\ 専門科目: [選択問題] 自作解答}
\author{解答作成者: \answerCreater}
\date{最終更新日: \today}

\lstset{
  basicstyle={\ttfamily},
  identifierstyle={\small},
  commentstyle={\smallitshape},
  keywordstyle={\small\bfseries},
  ndkeywordstyle={\small},
  stringstyle={\small\ttfamily},
  frame={tb},
  breaklines=true,
  columns=[l]{fullflexible},
  numbers=left,
  xrightmargin=0zw,
  xleftmargin=3zw,
  numberstyle={\scriptsize},
  stepnumber=1,
  numbersep=1zw,
  lineskip=-0.5ex
}

\begin{document}
\maketitle
\begin{abstract}
  
\end{abstract}
\tableofcontents
\lhead{電気通信大学 I専攻 \examYear 年度入試 専門科目: [選択問題] 自作解答}
\cfoot{\thepage}

\newpage
\section{電気回路}
\begin{enumerate}[(1)]
  \item 
\end{enumerate}

\newpage
\section{電磁気学}
\begin{enumerate}[(1)]
  \item 
\end{enumerate}

\newpage
\section{確率統計}
\begin{enumerate}[(1)]
  \item 
\end{enumerate}

\newpage
\section{信号処理}
\begin{enumerate}[(1)]
  \item 
\end{enumerate}

\newpage
\section{アルゴリズムとデータ構造}
\begin{enumerate}[(1)]
  \item 
\end{enumerate}

\newpage
\section{計算機の基本原理}
\begin{enumerate}[(1)]
  \item
\end{enumerate}

\newpage
\section{数値計算}
\begin{enumerate}[(1)]
  \item 
\end{enumerate}

\newpage
\section{離散数学とオートマトン}
\begin{enumerate}[(1)]
  \item 
\end{enumerate}


\end{document}
